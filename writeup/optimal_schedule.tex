\section{Optimal work and rest scheduling with fixed costs}

A worker chooses to work $h$ hours of work and then $r$ hours of rest.
They cycle back and forth between wage and rest.

The worker earns a wage $w$ but has convex costs, $c(h)$, with $c'(h) > 0$ and $c''(h) > 0$. 
The worker gets a benefit from rest, $b(r)$, with $b'(r) > 0$ and $b''(r) < 0$.
Starting a shift of worker costs $c_F$.

The worker's utility over a period $h + r$ is then
\begin{align}
  u = wh - c(h) - c_F + b(r). 
\end{align}
Let $u^*$ be the maximized value of that utility. 
The worker wants to maximize average utility, and so they maximize
\begin{align}
  \bar{u} = \max_{h,r} \frac{wh - c(h) - c_F + b(r)}{h + r}
\end{align} 
The first order conditions are
\begin{align}
 \frac{w - c'(h)}{u^*} - \frac{1}{h + r} = 0  
\end{align} 
and
\begin{align}
 \frac{b'(r)}{u^*} - \frac{1}{h + r} = 0  
\end{align} 
which implies that
\begin{align}
  w - c'(h^*) = b'(r^*)
\end{align} 
and that
\begin{align}
  w - c'(h^*) = b'(r^*) = \frac{u^*}{h + r}.
\end{align}

\subsection{When switching costs are high, there is less switching}
If $c_F$ goes up, $\bar{u}^*$ goes down, which implies that $r^*$ goes up, as $b(r)$ is concave.
Similarly, $h^*$ goes up.
When switching costs are high, there is less switching.

\subsection{Labor-Leisure trade-off}
When $w$ is higher, average utility has to go up, so $r^*$ goes down and $h^*$ goes up.
When labor becomes relatively more attractive, people worker for longer shifts. 

\subsection{Empirical question: does productivity fall?}

\subsection{What is the elasticity of labor supply?}

Hours worked as a fraction of time available is $\frac{h}{h + r}$.





